\documentclass[
	12pt, % font size
	parskip=half, % space between paragraphs instead of indentation
	digital, % digital or print
	twoside, % oneside / twoside
	openright, % openright for title page and subsequent chapters on right pages in twosided layout
]{bsc}
\usepackage{lipsum}
\usepackage{float}

\geometry{
	width=155mm,
%	bindingoffset=5mm,
}
\fancyhfoffset{0pt}

\title{Imaging of a degenerate Fermi gas of ultracold Lithium atoms}
\author{Robin Eberhard}
\date{April 1, 2016}

\usepackage{amsmath,amssymb}
\usepackage{esdiff} % derivatives
\usepackage{commath} % math macros
\usepackage{bbm} % blackboard style symbols
\usepackage{xfrac} % sfrac
\usepackage{mathtools} % coloneqq

% Links
\newcommand{\namedeqref}[2]{\hyperref[#2]{#1}~\eqref{#2}} % macro to reference equations by name
\renewcommand{\appref}[1]{\hyperref[#1]{Appendix~\ref{#1}}} % macro to print appendix references

% Equations
\newcommand{\eqpunct}[1]{\mathpunct{#1}} % embed punctuation in equations
\newcommand{\eqname}[2]{\label{#2} \\[-1em] \tag*{\llap{\emph{#1}}}} % print a title for the equation

% Images
\usepackage{pgf}
\newcommand{\plt}[3]{
	\begin{figure}[h]
		\begin{center}
			\input{plots/#1.pgf}
		\end{center}
		\caption[#2]{ \textbf{#2} \enskip #3}
		\label{fig:#1}
	\end{figure}
}
\newcommand{\draft}[3]{
\begin{figure}[h]
	\begin{center}
		\includegraphics{drafts/#1.pdf}
	\end{center}
	\caption[#2]{ \textbf{#2} \enskip #3}
	\label{fig:#1}
\end{figure}
}
% Abbreviations

% Math symbols
% TBD

% Units
\RequirePackage{siunitx}
\DeclareSIUnit\parsec{pc}

% Code
\newcommand{\mmainline}[1]{\texttt{#1}}

\begin{document}

% Front matter
\pagenumbering{Roman}

\input{content/titlepage}
\cleardoublepage
\thispagestyle{plain}

\makeatletter
\begin{center}
	\large\textbf{\@title}\\
	\normalsize\@author
\end{center}
\makeatother

\paragraph{Abstract}

This work describes the characterization and implementation of a new camera system for imaging of atomic density distributions in a multispecies ultracold atom experiment. The high quantum efficiency of the camera and its new readout mode, which allows for faster acquisition, are perfectly suited for scientific imaging. In this work, the noise sources that are induced from the camera itself, being the readout and dark noise, are described. To understand these, the setup of a typical CCD-camera is explained. At last, the imaging setup, including the magnification of an atomic cloud was tested on an ideal Fermi gas. The high resolution of the system and the noise reduction shows a precise distribution, which is explained and the degeneracy parameter is then extracted to be around $T/T_F=$ \todo{temperature...}.

\begin{otherlanguage}{ngerman}

\paragraph{Zusammenfassung}

\lipsum[1-1]

\end{otherlanguage}


\tableofcontents

% Main matter
\cleardoublepage\pagenumbering{arabic}

\chapter{Introduction}
\begin{itemize}
	\item Nothing yet...
\end{itemize}
\chapter{Setup for high resolution imaging}

\section{Experimental requirements}
\begin{itemize}
	\item Absorption imaging
	\item How CCDs work
	\item Cooling temperatures, timescales for imaging
\end{itemize}

\section{Camera for double species imaging}
\subsection{Comparison with the present setup}
\begin{itemize}
	\item Higher resolution
	\item Faster readout
	\item Higher quantum efficiency
	\item Less dark noise (to be confirmed)
\end{itemize}

\subsection{Dark current}
\begin{itemize}
	\item Theory on Dark current
	\item Temperature dependent
	\item Logarithmic dependency
	\item Water cooling
\end{itemize}
\plt{electrpp}{Dark noise}{
	The dark noise follows a power law dependency. Since these measurements were taken without water cooling installed, deviations are visible as the temperature reaches $\SI{-70}{\degreeCelsius}$. The convergence to zero on the counts and their variance indicates accurate imaging when low temperatures are used. Gain in this measurement was minimal and the exposure time set to $\SI{100}{\second}$, such that dark current was the dominant noise source.
}

\newpage
\subsection{Readout noise}
\begin{itemize}
	\item How does pixel shifting work? %http://www.mssl.ucl.ac.uk/www_detector/ccdgroup/optheory/ccdoperation.html
\end{itemize}
\plt{hvspeed}{Readout noise}{
	The pixels are shifted row-wise into the readout register, depending on the vertical shift speed ($v_{ss}$) and then moved pixel-by-pixel with the horizontal shift speed into the analogue to digital converter. Since noise reduction is important, minimal horizontal shift speeds will be used, while the vertical shift speed does not seem to affect the variance. To make the readout the dominant noise source, temperature was set to $\SI{-69}{\degreeCelsius}$ and exposure to $\SI{10}{\milli\second}$
}

\newpage
\subsection{Quantum efficiency}
\begin{itemize}
	\item Little bit of theory
	\item Reference to Carmens' thesis
\end{itemize}

\subsection{Pixel correlations}
\begin{itemize}
	\item Mainly the measurement (TBD)
	\item Some example images here maybe?
\end{itemize}

\section{Mechanical shutter}

\subsection{Electronic setup}
\begin{itemize}
	\item A simplified circuit
	\item Explanation of the parts
\end{itemize}

\newpage
\subsection{Dynamical properties}
\plt{shutterDiodeSignal}{
	Shutter characterization}{The dynamics of the shutter were measured using a laser with a variable horizontal offset, which is fixed in this plot, and a photodiode measuring the laser intensity. For various offsets, error functions were fitted yielding the time until the shutter opens to this offset.
}
\todo{appendix image}
\plt{shutterOpen}{Sample dynamics}{
	Opening velocity was measured using the beam diameter and the time the shutter needed to transverse it. It is noticable, that the opening velocity on the right side is faster at first than on the left side. This is due to the structure of the shutter, as can be seen in [Appendix image of shutter].
	The overall opening speed on the other hand is not affected by this and seems to be linear with the offset.
}

\newpage
\section{Mask for the CCD sensor}
\subsection{Fast kinetics mode}
\begin{itemize}
	\item Why it is good
	\item Shifting timescales
\end{itemize}

\newpage
\subsection{Frequency response of an imaging system}
\plt{slit_dist}{Distance dependant diffraction}{
	A slit was placed on a movable platform and diffraction was measured for various offsets. The diffraction frequency rises as the distance gets closer to the CCD. As soon as the frequency is of the order of one pixel, the diffraction is unnoticable, therefore higher frequencies, or closer slit positions are preferred.
	}
\plt{slit}{Diffraction measurement}{
	In order to characterize the diffraction on the CCD, a slit was placed as close as possible. The parameters were then measured as distance $d=\SI{10.9}{\milli\meter}$, opening $a=\SI{2.5}{\milli\meter}$ using a ruler, and wavelength $\lambda =\SI{852}{\nano\meter}$ from the laser specifications. The blue curve is the experimental data, while the red curve was fitted, leaving distance and opening free. They were found to be $d^\prime=\SI{11.0\pm0.3}{\milli\meter}$ and $a^\prime=\SI{2.470\pm0.013}{\milli\meter}$, which is in close agreement.
	}

\newpage
\subsection{Optimization of the masking setup}
\begin{itemize}
	\item Custom slit properties
\end{itemize}


\chapter{Thermometry of an ultracold ideal fermi gas}
\label{ch:idealfermigas}
The purpose of the camera is to measure scientifically important data from dense atomic clouds consisting of Lithium and Caesium.
The improvement of the resolution in the whole imaging setup now allows to explore new attributes that could not be measured before and as an example, ultracold ideal Fermi Gases were chosen. They are apparent at very low temperatures, where only few atoms due to evaporative cooling are present. Their density distribution differs slightly from a gaussian form and from that, one can extract temperatures and atom numbers.

In order to image the atoms, a technique called absorption imaging is used which is explained in the next section, before the introduction to ideal Fermi Gases.

\section{Absorption imaging}
In order to find microscopic attributes of atoms, or systems of atoms, it is necessary to look at the atoms themselves. This is commonly accomplished using either fluorescence or absorption imaging\cite{Murmann2011}. In both cases, a laser beam is pointed at an atomic cloud, that is cooled and confined in a trap. In fluorescence imaging, the scattered light is collected, typically in a direction that is different than the illuminating beam.
The intensity from the light through this method is not very high, since it is radiated in all directions. Therefore, long exposure times are required during which atoms can move and the information about the initial density and energy distribution is lost. Nevertheless, it is useful for single- and few-atom detection.

In contrast, in absorption imaging\cite{helmrich2013}, the transmitted intensity of the imaging beam is recorded. Without atoms, one would see a beam profile of the laser beam. With atoms, a shadow is visible due to the atoms "blocking" the light. This is accomplished, by correctly tuning the laser to a resonance frequency of the atoms, which enables them to absorb the light, exciting them to a higher state. Through spontaneous emission, the atoms will decay, making it possible to excite them once again. This method works well, when the "signal" from the absorbed light is significantly larger compared to the noise sources.

There are a set of optical elements in the imaging path, like lenses to collimate the image and refocus it, or mirrors to guide the light into the camera. Since the surfaces will most likely introduce errors into the imaging, for example from impurities or dust, only the absorption image will not suffice to gain reliable data. This is compensated by taking a total of three pictures, in order to extract only the relevant information from the image.

This can be understood when looking at the light intensity $I_{CCD}$ reaching the camera. The atom cloud has an optical density $OD$, therefore the intensity can be written as\cite{Murmann2011}
\begin{equation}
I_{CCD} = I_0 e^{-OD} + I_{back},
\end{equation}
where it decreases from the incident laser intensity $I_0$ due to light scattering by atoms. The intensity $I_{back}$ describes the background signal, that is found when the CCD is not being illuminated by a laser such as readout noise, dark noise or stray photon light. All the interesting attributes of atoms are found by looking at the optical density, therefore in order to extract that, a background frame is subtracted from the absorption image and the laser profile divided, leaving
\begin{equation}
\frac{I_{CCD} - I_{back}}{I_0} = e^{-OD}.
\end{equation}
The laser intensity $I_0$ is measured in a separate frame, containing the laser intensiy $I_0' = I_0 + I_{back}$ and also the background $I_{back}$. Finally, the equation yields
\begin{equation}
\frac{I_{CCD} - I_{back}}{I_0' - I_{back}} = e^{-OD}.
\end{equation}

From the resulting optical density, one can now conclude, for example, atom density distributions, atom numbers or excitation rates.

\section{Density distributions of ideal Fermi gases}
\label{sec:densdistrfermi}

Ideal Fermi gases offer a new aggregate, that is complementary to Bose-Einstein condensate. They are found on the Bardeen-Cooper-Schrieffer side of the Feshbach resonances as a polarized species with only one spin component. At very cold temperatures, they start to differ from a gaussian distribution which is further investigated in this chapter.

The distribution of the atoms depends on the fraction of their temperature to the Fermi temperature\cite{Ketterle2008} $T_F$. For $T/T_F \gg 1$, the atoms will follow a gaussian distribution, which can be identified using the gaussian radius:
\begin{equation}
\sigma _i = \sqrt{\frac{2k_BT}{mw_i^2}},
\end{equation}
with the mass of Lithium $m$ and the trapping frequency $w_i$ in the direction of the radius. Due to the alignment of the dipole trap, the cloud will not have a spherical shape and will therefore have different radii.

In the degenerate regime however, for $T/TF \ll 1$, the radius is described by the Fermi radius
\begin{equation}
R_{Fi} = \sqrt{\frac{2E_F}{mw_i^2}},
\end{equation}
using the Fermi energy $E_F$, due to the fermions filling up the eigenstates of the potential.

It is therefore suggested\cite{Ketterle2008} to use a unified radius, as the temperatures are not known a priori:
\begin{equation}
R_i^2 = \frac{2k_BT}{mw_i^2}f( e^{\frac{\mu}{k_BT}}).
\end{equation}
The interpolation function $f(x)$ is hereby:
\begin{equation}
f(x) = \frac{Li_1(-x)}{Li_0(-x)}
\end{equation}
where $Li_n$ is polylogarithm and can be defined as
\begin{equation}
Li_s(z) = \sum_{k=1}^{\infty} \frac{z^k}{k^s}.
\end{equation}

In our case, we integrate over all but one axes, and therefore find the fitting function for the atom numbers:
\begin{equation}
\label{eq:n1d}
n_{1D}(x) = n_{1D,0}\frac{Li_{5/2}\left( \pm \mathrm{exp}\left[ q-\frac{x^2}{R_x^2}f(e^q)\right] \right)}{Li_{5/2}(\pm e^q)}.
\end{equation}
The derivation can be found in \cite{Ketterle2008}. The parameter $q=\frac{\mu}{k_BT}$ can be extracted, which contains information about the chemical potential $\mu$ and the temperature $T$.

This parameter can then be used to calculate the degeneracy parameter:
\begin{equation}
\label{eq:tovertf}
\frac{T}{T_F} = \left[ -6 Li_3(-e^q) \right]^{-1/3}.
\end{equation}

To compensate for the finite resolution of the chip on the camera, the cloud can be imaged after a short time of flight $t$. The temperature can then be calculated from the dynamics as
\begin{equation}
\label{eq:temp}
k_BT = \frac{1}{2} mw_i^2 \frac{R_i^2}{1+w_i^2t^2}\frac{1}{f(e^q)}.
\end{equation}


	
\section{Finding properties of the Fermi gas}

As seen before, from an ideal Fermi gas, one can deduce the temperature and Fermi temperature of a gas.
In order to implement ideal Fermi gases, Lithium was prepared in an optical dipole trap. The power was ramped down several times at the Feshbach resonance $B=\SI{896}{\gauss}$, until only a fraction of the atoms remained. The spin-down atoms received a short laser pulse, so that the cloud only consisted of spin-up $^6$Li.

\pltCustom{
	\begin{center}
		\includegraphics[width=1\textwidth]{drafts/fermi_fit.pdf}
	\end{center}
	\begin{textblock}{2}(0.75,-4.7)
		\textbf{a.}
	\end{textblock}
	\begin{textblock}{2}(4.55,-4.7)
		\textbf{b.}
	\end{textblock}
	\begin{textblock}{2}(8.35,-4.7)
		\textbf{c.}
	\end{textblock}
}{fermi_fit}{Fitting the fermi distribution to an ideal Fermi gas}{In order to find the physical properties of a fermionic cloud, they were imaged at three different trap depths which correspond to the $\bar{\omega} = (\omega_x \omega_y \omega_z)^{1/3}$, where $w_i$ is the trap frequency in the corresponding axis. As of the writing of this thesis, the imaging system was not yet well aligned in the x-Direction, therefore the data does not represent the theory well. The results of the fits are given in Table \ref{tab:fermi_fit}.}

The measurement was executed for three different powers of the laser beam. The atoms were imaged after a short time of flight of \SI{1}{\milli\second}. In order to fit the function \refEq{n1d}, the absorption image yielding the optical density as seen in \refFig{fermi_fit} was integrated over one axis, giving a one dimensional distribution for the atoms.

As the imaging is not properly optimized at this point, the distribution in the horizontal (x-axis) direction does not represent the theory well, even after averaging over several acquisitions. This problem could not be resolved due to timing constrictions, therefore the fit parameters were extracted solely from the vertical direction.

On the vertical axis, a common systematic error was a small peak for high values, which did not correspond to the atoms and is probably due to the lenses, which are not properly aligned. This was excluded from the fit. Due to the alignment in the x-Direction, the measurement could not be included, as the fragments disturbed the data too much.
\begin{table}
	\begin{center}
		\begin{tabular}{M{1cm}!M{4cm}|M{4cm}|M{4cm}N}
			& $\bar{\omega}=\SI{53.8}{\hertz}$ & $\bar{\omega}=\SI{67.8}{\hertz}$ & $\bar{\omega}=\SI{119.8}{\hertz}$ & \\[7pt]
			\thickhline
			$n_{1D}$ & $25.6\pm0.1$ & $34.07\pm0.13$ & $44.98\pm0.24$ & \\ [7pt]
			\hline
			$q$ & $1.87\pm0.35$ & $1.08\pm0.34$ & $-2.0\pm1.85$ & \\[7pt]
			\hline
			$R_y$ & $23.9\pm0.72$ & $23.80\pm0.75$ & $26.79\pm1.06$ & \\[7pt]
			\hline
			$T$ & $(5.92\pm0.55)*10^{-8}$ & $(7.43\pm0.14)*10^{-7}$ & $(1.23\pm0.25)*10^{-6}$ & \\[7pt]
			\hline
			$T/T_F$ & $0.34$ & $0.42$ & $1.01$ & \\[7pt]
		\end{tabular}
	\end{center}
	\setCaption{Fit parameters for various trap depths}{The Fermions were fitted using \refEq{n1d}, where the parameter $n_{1D}$ describes the amplitude of the peak. $q$ is a shape parameter, which is negative if the cloud is thermal or positive if the cloud is degenerate. $R_y$ is then the clouds radius similar as the radius can be described in a gaussian distribution which is in units of pixels and can be calculated in natural units using the pixel size (\SI{13}{\micro\meter}) and the magnification ($7.5$).}
	\label{tab:fermi_fit}
\end{table}

In order to properly fit \refEq{n1d}, a linear function had to be added, as there was a constrant gradient on the chip, most likely due to the readout of the image. From the fit parameters in Table \ref{tab:fermi_fit}, it was possible to calculate the temperature and the degeneracy parameter from \ref{eq:temp} and \ref{eq:tovertf} respectively. From the results of $T/T_F$ it can be seen, that the gas is becoming more degenerate, as the trapping frequencies become lower. Additionally, the temperature of the gas is becoming cooler, as more atoms leave the trap, which is expected from evaporative cooling as well.

This experiment already shows the operation of the camera as a scientific instrument. As predicted from theory \cite{Ketterle2008}, the gas follows a Fermi distribution as the cloud becomes colder.
\chapter{Conclusion and outlook}

In this thesis, the new camera setup was implemented, which increases the resolution of the setup significantly. The first part described the imaging, where an atomic cloud was illuminated by laser beam. We saw, that it is important to have a lens with a low focal length to collimate the image of the cloud, in order to gain a high numerical aperture, which relates to the resolution of the system.

The Andor iKon M camera was introduced and the mechanics behind a CCD chip were explained, leading to an understanding of various noise sources. As the pixels in a camera are made of semiconductors, we found that a common noise source is the so-called dark noise, where thermal electrons are excited to the valence band, creating counts. This can be prevented by cooling the chip down. We found that at \SI{-70}{\degreeCelsius}, the dark current was already reduced to less than 40 electrons per pixel and second. Another common noise source was found to be the readout noise, which accumulates by shifting pixels into the readout register and the ADC.
As the readout is dependent on the vertical and horizontal shift speed, we found that the noise does not depend on the vertical shift speed. Therefore we concluded, that during measurements the horizontal shift speed has to be low, while the vertical shift speed can be high in order to allow for fast acquisition of multiple images as they are read out using the fast kinetics mode.

Using this acquisition mode, it is possible to acquire images without reading them out in between, as they are hidden behind a cover. This is an important feature of the camera, as the group is using a mixture of $^6$Li and $^{133}$Cs, which need to imaged consecutively as fast as possible.

In the end, the camera was tested on an ideal Fermi gas. We have seen that the distribution is gaussian like for $T/T_F > 1$ while it has the Fermi-distribution for $T/T_F < 1$. For several trap depths, the degeneracy parameters were found to be $(T/T_F)_1 = 0.34$, $(T/T_F)_2 = 0.42$ and $(T/T_F)_3 = 1.01$.

The detail of this distribution already shows the potential of the imaging system. Building on that, one could now also create a Fermionic superfluid from Lithium, using a spin imbalance. This new aggregate has a unique structure, where there is an unpolarized core surrounded by the majority spin component\cite{Shin2006,Zwierlein2006b}. The latter is then as well described by the same formulas as the polarized Fermi gas, which was also discussed in this thesis.

Due to the high resolution of the imaging system it is also possible to investigate Bubble polarons. Due to the repulsive interaction inside the mixture, it is possible to identify a hole in the distribution of the atoms, which would have been too small for the old imaging system.

\appendix
\chapter{Acquisition sequence}
In order to take the images, some bounds have to be considered. In detail, the acquisition is limited by the shift speeds.
For the fast kinetics mode, the sequence looks like this
\chapter{Testing software}

\listoffigures

\bibliography{literature/Mixtures.bib}{}
\bibliographystyle{unsrt}

\chapter*{Danksagung}
\pagenumbering{gobble}
Ich möchte an dieser Stelle die Gelegenheit nutzen, mich bei Prof. Dr. Matthias Weidemüller zu bedanken, der es mir ermöglicht hat in dieser Arbeitsgruppe meine Bachelorarbeit abzuschließen. Meinen herzlichsten Dank widme ich des weiteren Dr. Juris Ulmanis. Mit seiner Begeisterung für die Physik konnte er mich für die Welt der ultrakalten Quantengase gewinnen und mit seiner vielen Zeit konnte er aus jedem Problem interessante Physik gestalten. Des weiteren bedanke ich mich außerdem bei Stephan Häfner, der mit seinem physikalischen Geschick Fehler schnell entdecken konnte und mir somit Zeit und Frustration erspart hat. Mit Manuel Gerken konnte ich jeder physikalischen Fragestellung, ob einfach oder komplex, auf den Grund gehen und durch ihn wurden meine Stunden im Büro erst unterhaltsam. Ich habe die Zeit in dieser Arbeitsgruppe sehr genossen, dafür gebührt mein herzlichsten Dank.

\input{content/declaration}

\end{document}