\chapter{Conclusion and outlook}

In this thesis, a new camera setup was implemented, which increases significantly the resolution of the present absorption imaging. The Andor iKon M camera was introduced and the mechanics behind a CCD chip were explained, leading to an understanding of various noise sources. As the pixels in a camera are made of semiconductors, we found that a common noise source is the so-called dark noise, where thermal electrons are excited to the valence band, creating counts. This can be prevented by cooling the chip down. We found that at \SI{-70}{\degreeCelsius}, the dark current was already reduced to less than 4 electrons per pixel and second. Another common noise source was found to be the readout noise, which accumulates by shifting pixels into the readout register and the ADC.
As the readout is dependent on the vertical and horizontal shift speed, we found that the noise does not depend on the vertical shift speed. Therefore we concluded, that during measurements the horizontal shift speed has to be low, while the vertical shift speed can be high in order to allow for fast acquisition of multiple images as they are read out using the fast kinetics mode.

Using this acquisition mode, it is possible to acquire images without reading them out in between. This is done by illuminating only a part of the chip. When the exposure is finished, the illuminated part is shifted down behind a cover, so that the top part is ready for the next image. This is an important feature of the camera, as the group is using a mixture of $^6$Li and $^{133}$Cs, which need to imaged consecutively as fast as possible.

In the end, the new imaging system was tested on an ideal Fermi gas. Fermionic Lithium atoms were prepared in an optical dipole trap with only one spin component. We observed that the distribution of the atoms follows a Fermi-distribution for $T/T_F < 1$. For several trap depths, the degeneracy parameters were found to be $(T/T_F)_1 = 0.34\pm0.26$, $(T/T_F)_2 = 0.42\pm0.23$ and $(T/T_F)_3 = 1.01\pm0.56$.

From the details of the distributions due to the high resolution of the imaging setup, we were able to distinguish the density profile of the fermionic cloud from a gaussian shape, which already shows the potential of the imaging system. Building on that, one could now also create a Fermionic superfluids, using a spin mixture. This state has a unique structure, where there is an unpolarized core surrounded by the majority spin component\cite{Shin2006,Zwierlein2006b}. The latter is then as well described by the same formulas as the polarized Fermi gas, which was also discussed in this thesis. This would be an exciting experiment as it will further put the resolution of the imaging system to a test.
