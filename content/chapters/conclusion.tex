\chapter{Conclusion and outlook}

In this thesis, the new camera setup was implemented, which increases the resolution of the setup significantly. The first part described the imaging, where an atomic cloud was illuminated by laser beam. We saw, that it is important to have a lens with a low focal length to collimate the image of the cloud, in order to gain a high numerical aperture, which relates to the resolution of the system.

The Andor iKon M camera was introduced and the mechanics behind a CCD chip were explained, leading to an understanding of various noise sources. As the pixels in a camera are made of semiconductors, we found that a common noise source is the so-called dark noise, where thermal electrons are excited to the valence band, creating counts. This can be prevented by cooling the chip down. We found that at \SI{-70}{\degreeCelsius}, the dark current was already reduced to less than 40 electrons per pixel and second. Another common noise source was found to be the readout noise, which accumulates by shifting pixels into the readout register and the ADC.
As the readout is dependent on the vertical and horizontal shift speed, we found that the noise does not depend on the vertical shift speed. Therefore we concluded, that during measurements the horizontal shift speed has to be low, while the vertical shift speed can be high in order to allow for fast acquisition of multiple images as they are read out using the fast kinetics mode.

Using this acquisition mode, it is possible to acquire images without reading them out in between, as they are hidden behind a cover. This is an important feature of the camera, as the group is using a mixture of $^6$Li and $^{133}$Cs, which need to imaged consecutively as fast as possible.

In the end, the camera was tested on an ideal Fermi gas. We have seen that the distribution is gaussian like for $T/T_F > 1$ while it has the Fermi-distribution for $T/T_F < 1$. For several trap depths, the degeneracy parameters were found to be $(T/T_F)_1 = 0.34$, $(T/T_F)_2 = 0.42$ and $(T/T_F)_3 = 1.1$.

The detail of this distribution already shows the potential of the imaging system. Building on that, one could now also create a Fermionic superfluid from Lithium, using a spin imbalance. This new aggregate has a unique structure, where there is an unpolarized core surrounded by the majority spin component\todo{cite some superfluid paper}. The latter is then as well described by the same formulas as the polarized Fermi gas, which was also discussed in this thesis.

Due to the high resolution of the imaging system it is also possible to investigate Bubble polarons. Due to the repulsive interaction inside the mixture, it is possible to identify a hole in the distribution of the atoms, which would have been too small for the old imaging system.
