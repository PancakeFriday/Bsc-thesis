\chapter{Introduction}
Conducting experiments, measuring attributes and taking data are quantities that go hand in hand in scientific fields. Already in the early 1900s, imaging systems were chosen as data acquisition in astronomy. A complete optical system allowed for magnification of planets, our sun or nebulae. This trend started with the use of analogous photo plates \cite{hdaplates}, while nowadays so-called charge-coupled devices (CCD) or complementary metal-oxide-semiconductors (CMOS) are used as they can take the data digitally.

This also found application in the field of ultracold quantum physics --- an experimental field that is around since the 1970s. Hereby, atoms are prepared in traps in order to reduce their velocities. The atoms can then be detected by illuminating them with an imaging beam, a laser beam that is resonant with the atoms. The atoms will cast a shadow, which can be focused onto the chip, where the result is digitalized and can be analyzed. This process offers some challenges. At low temperatures, the trap width can not be arbitrarily big. This means, that only fewer atoms can be imaged, therefore a high resolution imaging is required in order to distinguish them from the background noise. But the resolution of the system is not only limited by the chip and pixel size of the camera, it also needs to be refined on optical elements such as lenses and mirrors.

The experiment at this group uses a mixture of $^6$Li and $^{133}$Cs, in order to measure for example the Efimov effect, ideal Fermi gases or Polarons. An important part in this setup is the acquisition, where the species need to be acquired separately.
A new system  now allows to image both $^6$Li and $^{133}$Cs only after a short delay, instead of taking the images from different axes or clouds.

This thesis builds upon the optical system, that made the high resolution possible and will introduce a new camera into the experiment, which has new features, that allow for noise reduction and an improvement in acquisition timings.