\chapter{Introduction}
Conducting experiments, measuring physical quantities and taking data are features that go hand in hand in scientific fields. Already in the early 1900s, imaging systems were chosen for data acquisition in astronomy. A complete optical system allowed for magnification of planets, our sun or nebulae. Already then, they were able to store their acquisitions on photo plates \cite{hdaplates}. Nowadays so-called charge-coupled devices (CCD) or complementary metal-oxide-semiconductors (CMOS) are used as they can take the data digitally.

This also found application in the area of ultracold quantum physics --- an experimental field that started around 1970s. In these experiments, atoms are prepared in traps at very low velocities, corresponding to the temperature regime around \SI{1}{\milli\kelvin} and below. The atoms can then be detected by illuminating them with a laser beam, that is resonant with the atoms. The atoms will cast a shadow, which can be focused onto the chip, where the result is digitalized and can be analysed. Interesting physical quantities of atomic clouds are usually density distributions and atom numbers, since they contain a lot of information about various properties of the ultracold gas, for example, temperature, density or compressibility. In order to calculate atom numbers, a high resolution imaging system is not needed, since the optical density is extracted from the whole chip and not individual pixels. Therefore if the magnification of the system is known, one can directly conclude the atom number.

In order to directly probe interactions between atoms and the resulting distributions, an optical imaging system with a high resolution is commonly used. It consists of several lenses that are carefully set up in order to gain optimal resolution. A high number of pixels on the chip and high quantum efficiencies then allows to extract detailed information about the cloud. Especially for a small number of atoms it is additionally important to have small noise in the camera as well as a high sensitivity to photons.

Our experiment uses an ultracold mixture of $^6$Li and $^{133}$Cs atoms. This combination of atoms is particularly well suited for the study of the Efimov effect \cite{Pires2014,Ulmanis2015,Ulmanis2016} and different polarons\cite{Blinova2013,Massignan2014,Devreese2009} (Fermi-polaron, Bose-polaron, Bubble-polaron). The interactions between Li and Cs atoms, that can be tuned by broad Feshbach resonances, will allow to explore various few- and many-body effects. The study of all of these phenomena will require a well-tuned imaging system.

This thesis describes a double-species optical imaging system with high resolution that was adapted and built into the experimental apparatus. The setup features a new CCD camera with reduced noise and high quantum efficiency. A novel acquisition mode allows to drastically improve the acquisition timings. This thesis is structured as follows.

At first, the imaging setup is explained in \refCh{expreq}. \refCh{camera} then introduces the new camera used and highlights its qualities and improvements to the old system. In order to fully understand the noise properties of the chip, the basics of a CCD detector are explained, followed by measurements of different noise sources, for example the dark current, and the readout noise.
Finally an overview about the quantum efficiency and how it affects the imaging quality is given.

A mechanical shutter was built into the imaging path. The shutter can be electronically controlled, in order to do that, a circuit was implemented in Section \ref{subsec:shutter_electronic}. The dynamics of the shutter were investigated to extract accurate timings of the opening and closing of the shutter.

\refSec{maskccd} focuses on the new acquisition mode of the Andor camera, which requires to block parts of the chip. In order to do that, a slit was built and placed in front of the chip. The diffraction introduced by the slit was characterized and explained theoretically.

In \refCh{idealfermigas} the imaging of an ideal Fermi gas is described. The density distribution of an ultracold cloud of harmonically trapped atoms is briefly explained in \ref{sec:densdistrfermi}, the chapter concludes with measurements and analysis of temperature of an ideal Fermi gas of Li atoms.