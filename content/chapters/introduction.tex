\chapter{Introduction}
Conducting experiments, measuring attributes and taking data are quantities that go hand in hand in scientific fields. Already in the early 1900s, imaging systems were chosen as data acquisition in astronomy. A complete optical system allowed for magnification of planets, our sun or nebulae. Already then, they were able to store their acquisitions on photo plates \cite{hdaplates}, while nowadays so-called charge-coupled devices (CCD) or complementary metal-oxide-semiconductors (CMOS) are used as they can take the data digitally.

This also found application in the field of ultracold quantum physics --- an experimental field that is around since the 1970s. Hereby, atoms are prepared in traps in order to reduce their velocities. The atoms can then be detected by illuminating them with an imaging beam, a laser beam that is resonant with the atoms. The atoms will cast a shadow, which can be focused onto the chip, where the result is digitalized and can be analyzed. Interesting attributes of atomic clouds are usually density distributions and atom numbers. For the latter, a high resolution imaging system is not needed, since the optical density is extracted from the whole chip and not individual pixels. Therefore if the magnification of the system is known, one can directly conclude the atom number.

In order to directly see interactions between atoms and their distributions they yield, a high resolution system is commonly used. Therefore a system of lenses needs to be carefully set up in order to gain maximum resolution. A high number of pixels on the chip then allows to see detailed information of the cloud. Especially for small number of atoms it is additionally important to have small noise in the camera as well as a high sensitivity to photons, being the quantum efficiency.

Interesting features were already observed for example in fermionic superfluids, where a gas of spin imbalanced Fermions were prepared. The cloud was "stirred" using a laser beam, revealing small vortices inside the cloud which were visible as holes.

Our experiment uses a mixture of $^6$Li and $^{133}$Cs. The new imaging system is particularly well suited for the study of the Efimov effect \cite{Pires2014,Ulmanis2015,Ulmanis2016} and different polarons (Fermi-polaron, Bose-polaron, Bubble-polaron). In case for the Bubble polaron, the resolution of the imaging system will allow to find a hole in the density distribution of the atoms.

This thesis builds upon the optical system, that made the high resolution possible and will introduce a new camera into the experiment, which has new features, that allow for noise reduction and an improvement in acquisition timings. At first, the imaging setup is explained in \refCh{expreq}. \refCh{camera} then introduces the camera used and will highlight its qualities and improvements to the system\ref{sec:comparison}. In order to fully understand noise accumulation on the chip, the setup of a CCD is then explained in \refCh{ccd_basics}, followed by different noise sources being the dark current\ref{subsec:darkcurrent}, which is a temperature dependent noise that will accumulate over time and the readout noise\ref{subsec:readoutnoise}, which adds noise as the image is being read out.
Subsection \ref{ch:quantumeff} will then give an overview about the quantum efficiency and how it affects the imaging quality.

In \refCh{idealfermigas} a measurement was concluded in order to test the camera. An ideal Fermi gas was prepared, which shows a unique distribution different from Bose-Einstein condensates, from which the temperature and Fermi temperature were concluded.