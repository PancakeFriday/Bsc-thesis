\chapter{Testing the camera: Superfluids}
\begin{itemize}
	\item Theory on superfluids
	\item Spin population profile
	\item Measurement
\end{itemize}
The purpose of the camera is to measure scientifically important data from dense atomic clouds consisting of Lithium and Caesium.
The improvement of the resolution in the whole imaging setup now allows to find new attributes that could not be measured before and as an example, fermionic superfluids were chosen. The key signature of a superfluid transition can be found by creating a population imbalance, that then shows a unique structure in the centre of the cloud, differing them from the Thomas-Fermi distribution. \todo{check this}

\begin{itemize}
	\item History on first measurements
	\item macroscopic features
	\item measurements of other groups
	\item own measurements (maybe comparison)
\end{itemize}

Superfluidity can be observed macroscopically for example in cooled Helium, where the fluid has no viscosity. It will also defy gravity, climbing walls and ignore surface tension. This is still mysterious in parts, therefore in order to understand this better, an ultracold quantum system can be used as a model.

% http://www.nature.com/nature/journal/v435/n7045/pdf/nature03858.pdf
The first confident detection were carried out by Zwierlein et al in 2005 \todo{cite zwierlein see comment}. A Bose-Einstein condensate was prepared using evaporative cooling and a dipole trap. An additional laser beam, that is split using an acousto-optic deflector into two beams, that are separated by $3/4$th of the trap diameter will then be used to "stir" the Lithium spin mixture.

This laser is merely used to prove, that a superfluid has been prepared. When the correct feshbach resonances are applied, the condensate will start to create vortex lattices, that are visible in the cloud as holes. In an optimal setup, the group was able to find up to 40 vortices in a single atomic cloud.

This laid the ground stone for the detection of superfluids, as no other effect is known to cause these vortices.
Since it was possible from then on to prepare superfluids confidently, future research concentrated on understanding this exotic aggregate state.

The group of ... \todo{...} took a look at when the superfluid transition takes place, while also introducing another detection method. When a superfluid is prepared with a spin imbalance, for example 70\% majority and 30\% minority spin components (which could be either spin up or spin down), a center core forms, in which all the minority components are apparent, together with the same amount of majority spins. The wings on the outside of the core is then only populated by the remaining majority.

At this point, one can look at the Bose-Einsten condensate (BEC) and Bardeen-Cooper-Schrieffer (BCS) side of the Feshbach resonance, which is characterized using the interaction parameter $\frac{1}{k_Fa} > 0$ and $\frac{1}{k_Fa} < 0$ for BEC and BCS respectively.

The superfluid core is apparent in both cases, but more visible in the BEC case.