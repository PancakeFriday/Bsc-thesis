\chapter{Setup for high resolution imaging}

\section{Experimental requirements}

\section{Camera for double species imaging}
\subsection{Comparison with the present setup}

\subsection{Dark current}
\plt{electrpp}{Dark noise}{The dark noise follows a power- law dependency in the high temperature regime. Since these measurements were taken without water cooling installed, deviations are visible in the low temperature parts of the plot.}

\subsection{Readout noise}
\plt{hvspeed}{}{}

\subsection{Quantum efficiency}

\subsection{Pixel correlations}

\section{Mechanical shutter}
\plt{shutterDiodeSignal}{Shutter characterization}{The dynamics of the shutter were measured using a laser with a variable x offset, and a photodiode measuring the laser intensity. Fitting the signal using an error function yields the time until the shutter opened to this offset.}
\plt{shutterOpen}{Shutter characterization}{

}

\subsection{Electronic setup}

\subsection{Dynamical properties}


\section{Mask for the CCD sensor}
\subsection{Fast kinetics mode}

\subsection{Frequency response of an imaging system}
\plt{slit}{Diffraction measurement}{The characterized the diffraction effect on the camera using a custom-built slit, placing it as close as possible to the chip. The parameters were measured as $d=\SI{10.9}{\milli\meter}$, $a=\SI{2.5}{\milli\meter}$ and $\lambda =\SI{852}{\nano\meter}$}.
\plt{slit_dist}{}{}

\subsection{Optimization of the masking setup}

