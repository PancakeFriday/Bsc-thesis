\cleardoublepage
\thispagestyle{plain}

\makeatletter
\begin{center}
	\large\textbf{\@title}\\
	\normalsize\@author
\end{center}
\makeatother

\paragraph{Abstract}
In this work, a double-species optical imaging system with high resolution was adapted and built into an ultracold mixture apparatus. The high quantum efficiency of the camera and its new readout mode, which allows for faster acquisition, are perfectly suited for scientific imaging of atomic density distributions. During this thesis, the electronic noise of the CCD detector was extensively characterized and a mechanical shutter was set up. The complete imaging setup was used to measure the density distribution of an ideal Fermi gas of ultracold Lithium atoms. Derivations from a thermal distribution on the transition in the degenerate regime with the minimal achieved $T/T_F=0.34$ could be observed.


\begin{otherlanguage}{ngerman}

\paragraph{Zusammenfassung}
In dieser Arbeit wurde auf einem optischem Imaging System mit einer hohen Auflösung aufgebaut, welches anschließend in ein Experiment mit ultrakalten zweiatomigen Gasen eingebaut wurde. Mit der hohen Quanteneffizienz und dem neuen Auslesemodus der Kamera ist diese perfekt für Wissenschaftliches Imaging von atomaren Dichteverteilungen geeignet. Während dieser Arbeit wurde das elektrische Rauschen des CCD Detektors intensiv charakterisiert und ein mechanische Shutter wurde aufgebaut. Das gesamte Imagingsystem wurde dann benutzt, um Dichteverteilungen eines idealen Fermigases ultrakalter Litium Atome zu messen. Abweichungen von einer thermischen Verteileung im Übergang zum degenerierten Regime wurden mit dem minimalen $T/T_F=0.34$ beobachtet.

\end{otherlanguage}
